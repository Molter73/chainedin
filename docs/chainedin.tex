\documentclass[a4paper]{article}

\usepackage{titling}
\usepackage{fullpage}
\usepackage{fontenc}
\usepackage{listings}
\usepackage{mathptmx}
\usepackage{hyperref}
\usepackage{tikz}
\usepackage{graphicx}
\usepackage{float}
\usepackage{parskip}
\usepackage[none]{hyphenat}

\renewcommand{\contentsname}{Contenidos}

\hypersetup{
    colorlinks=true,
    linkcolor=blue,
    filecolor=magenta,
    urlcolor=blue,
}

\urlstyle{same}

\begin{document}

\title{ChainedIn}
\author{Moltrasio, Mauro Ezequiel}
\date{}
\renewcommand{\abstractname}{\vspace{-\baselineskip}}

\begin{titlingpage}
    \maketitle
    \begin{abstract}
        Clase: Desarrollo Web y de Apps
    \end{abstract}
\end{titlingpage}

\tableofcontents

\section{Introducción}
En este documento se detallará el proceso de diseño y desarrollo de un
sitio web para búsqueda de empleo. El sitio en cuestión se desplegará en
un entorno web, por lo que se deberá desarrollar un backend que permita la
persistencia y recuperación de datos, y un frontend capaz de mostrar esta
información de una manera amigable a usuarios.

\section{Descripción del problema}

El sitio en cuestión permitirá:
\begin{itemize}
    \item Crear nuevos usuarios con un rol específico (usuario o reclutador).
    \item Para reclutadores:
        \begin{itemize}
            \item Crear anuncios para ofertas de empleo.
            \item Recuperar la lista de empleos propios creados.
            \item Visualizar perfiles de candidatos a las distintas ofertas.
        \end{itemize}
    \item Para usuarios:
        \begin{itemize}
            \item Visualizar todas las ofertas de empleo disponibles.
            \item Postularse a distintas ofertas.
            \item Obtener una lista de las ofertas a las que se encuentra postulado.
            \item Editar su perfil.
        \end{itemize}
\end{itemize}

Podemos convertir los puntos previos en diagramas de caso de uso para tener
una mejor visualización:

\begin{figure}[H]
    \centering
    \scalebox{.6}{
        \input{build/uml.tex}
    }
\end{figure}
\begin{figure}[H]
    \centering
    \scalebox{.6}{
        \input{build/uml_001.tex}
    }
\end{figure}

Con estos diagramas de caso de uso, podemos ver que se tendrán dos tipos de
usuario, como se detallaba en la sección anterior y que ambos deberán poder
interactuar con los empleos disponibles pero de formas distintas. Así también
está claro que ambas partes pueden visuarlizar perfiles de usuarios, pero
sólo el reclutador puede ver los perfiles de otros usuarios.

\section{Solución propuesta}
Se propone implementar un sitio web que permita cumplir con los requisitos
expuestos en los diagramas de caso de uso del punto anterior. Siguiendo
las tendencias del desarrollo web moderno, se escribirá un frontend basado en
HTML, CSS y JavaScript, apoyándose en el framework Bootstrap para simplificar
la estructuración del sitio y aprovechando las capacidades responsivas del
mismo. Para el backend, se utilizará MySQL para la persistencia de datos y
PHP para el acceso a la misma y agregar la lógica de negocio adicional que
fuese necesaria.

\section{Arquitectura}
\subsection{Base de datos}
Empezaremos por el almacenamiento de datos. Se utilizará una base de datos SQL,
por lo que podemos comenzar por plantear una distribución de tablas como la
que se muestra a continuación:

\includegraphics[scale=0.65]{imgs/db.png}

Ahondando un poco en los detalles, cada usuario tendrá una entrada en la tabla
users, la cual se utiliza para el ingreso en el sitio, por lo que tendrá el
email del usuario como clave y su password hasheado. En esta tabla también se
especificará el tipo de usuario que tenemos (reclutador o usuario) con un
entero. Finalmente, cada usuario tendrá un id numérico que se utilizará para
identificarlo de forma más sencilla.

Asociado a cada usuario habrá también una entrada en la tabla profiles, la
cual almacena detalles como nombres, apellidos, experiencia laboral, etc. Esta
separación se realizó porque estos detalles en general sólo son necesarios
cuando se consulta el perfil de usuario y no nos es útil en la mayor parte de
la navegación del sitio, por lo que esta separación simplificaba las
consultas a la DB.

Se creará también una tabla para las vacantes de empleos llamada jobs. En la
misma se guardarán el título del empleo, una descripción, la ubicación de un
logo de la empresa que publica la oferta y la fecha de creación de la misma.
Vemos también que se almacena la id de usuario del reclutador que creó la
oferta, esto nos permitirá devolver las ofertas asociadas a un reclutador
específico, para que este no deba navegar en la totalidad de empleos para
llegar a sus vacantes.

Finalmente, para llevar un registro de postulaciones, la tabla applicants
guardará una lista que correlaciona ids de usuarios con ids de ofertas. Esto
nos permite realizar uniones entre las tablas jobs, users y applicants para
obtener distintos tipos de listados (postulantes a una oferta, ofertas a las
que un usuario se ha postulado, etc.).

\subsection{API}
Para controlar el acceso a la base de datos, se utilizó PHP como intermediario
para aplicar la lógica de negocio necesaria. Si bien PHP permite el renderizado
de HTML, se optó por tener una separación entre el backend y el frontend lo más
marcada posible, por lo tanto, los scripts de PHP sólo devuelven strings JSON
con la información solicitada. También se optó por hacer que el backend carezca
de estado, manteniendo la menor cantidad de información posible en la sesión de
PHP, sólo se guarda el id y el tipo de usuario, estos datos son suficientes
para validar que las peticiones provienen de un usuario autenticado y realizar
algunos ajustes de comportamiento según si el usuario es reclutador o no.

A continuación podemos ver el diagrama de secuencia de un ejemplo de petición a
la API:
\begin{figure}[H]
    \centering
    \scalebox{.6}{
        \input{build/uml_002.tex}
    }
\end{figure}

Los scripts principales del backend son los siguientes:
\begin{itemize}
    \item signup.php:
        \begin{itemize}
            \item POST: Permite la creación de usuario junto con datos de perfil.
        \end{itemize}
    \item login.php:
        \begin{itemize}
            \item POST: Realiza el inicio de sesión para el email y
                contraseña provistos.
            \item GET: Verifica si la petición tiene una sesión válida iniciada.
        \end{itemize}
    \item jobs.php:
        \begin{itemize}
            \item POST: Permite la creación de nuevas vacantes.
            \item GET: Obtiene vacantes abiertas, paginadas y ordenadas por
                fecha de creación. Si el usuario es un reclutador, se limitan
                los resultados a posiciones creadas por el reclutador. Si se
                provee un id, devuelve sólo los datos de la oferta con ese id.
        \end{itemize}
    \item profile.php:
        \begin{itemize}
            \item POST: Permite la actualización de datos del perfil.
            \item GET: Devuelve la información del perfil solicitado.
        \end{itemize}
    \item apply.php:
        \begin{itemize}
            \item POST: Permite que un usuario presente su candidatura a una
                oferta.
        \end{itemize}
\end{itemize}

\subsection{Frontend}

La parte final de la solución consiste en un frontend implementado con HTML,
CSS y JavaScript. Estas tres tecnologías están fuertemente establecidas como el
estándar para generación de páginas web. Para simplificar la tarea de estilizar
las páginas y volverlas responsivas, se decidió utilizar Bootstrap en su
versión 5.0. Inicialmente se evalúo la utilización de JQuery como librería
adicional, pero fue descartado en favor de utilizar JavaScript directamente,
ya que en sus versiones más recientes se ha simplificado mucho su uso y las
tareas a realizar no requerían de las abstracciones de JQuery.

Las páginas que se desarrollaron son:

\begin{itemize}
    \item Login: Permite el inicio de sesión de usuarios.
    \item Sign Up: Permite la creación de nuevos usuarios.
    \item Listado Ofertas: Muestra las vacantes de trabajos a usuarios. En caso
        de ser el usuario un reclutador, se muestran sólo las ofertas generadas
        por el mismo.
    \item Oferta: Muestra los detalles de una oferta y permite a usuarios el
        postularse a la misma. Para reclutadores, muestra una lista de
        candidatos.
    \item Perfil: Muestra detalles del perfil de un usuario. En caso de ser un
        usuario normal viendo su propio perfil, se muestra una lista de
        ofertas a las que se ha postulado.
    \item Edición de perfil: Permite editar detalles de un perfil de usuario.
\end{itemize}

Cabe destacar que en los casos de páginas con funcionalidades mixtas, basadas
en diferencias como tipo de usuario y coincidencia o no del id de usuario, la
lógica del frontend es manejada por el backend. Por ejemplo, cuando decimos
que en la página de ofertas se muestran todas las ofertas para usuarios o las
creadas por un reclutador para el reclutador, es el backend el que decide qué
ofertas se deben enviar, el frontend puede o no tener lógica particular para
manejar alguno de estos casos, pero siempre es el backend el que ejerce la
lógica de negocio y decide qué operaciones están o no admitidas en cada
situación. Esto se realizó de esta manera porque el frontend termina siendo
una forma "cómoda" para usuarios de realizar consultas HTTP, usuarios más
experimentados podrían utilizar las herramientas de desarrollador de los
navegadores para mostrar botones a los que no deberían tener acceso o hacer uso
de utilidades como cURL o PostMan para consultar al backend directamente, por
lo que es este último el que debe tomar estas decisiones.

\section{Conclusiones}
Mediante el uso de tecnologías establecidas y probadas, se ha podido
implementar una solución funcional y robusta al problema de crear una web de
búsqueda de empleos. Si bien en el mercado hay tecnologías más nuevas como ser
React.js para el frontend o la posibilidad de utilizar lenguajes como Python o
Go para el backend, HTMl, CSS, JavaScript, PHP y MySQL aún hoy demuestran ser
buenas alternativas para el prototipado rápido de aplicaciones y las librerías
como Bootstrap simplifican mucho la tarea de desarrolladores con menos
experiencia.

Si bien la solución final implementada no es perfecta, sería un buen punto de
partida para el desarrollo de una web más compleja y productiva.

\end{document}
